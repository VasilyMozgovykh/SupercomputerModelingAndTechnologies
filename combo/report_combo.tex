\documentclass[12pt]{article}

\usepackage[%
    includehead, includefoot,
    paperheight=290mm,
    paperwidth=205mm,
    textwidth=170mm,
    textheight=259mm,%
    left=17mm,%
    top=11mm,%
    right=18mm,%
    bottom=20mm,%
    headsep=0.5cm,%
    headheight=14pt,%
    footskip=0.0cm]{geometry}

\usepackage[utf8]{inputenc}
\usepackage[russian]{babel}
\usepackage{amsmath}
\usepackage{mathtools}

\begin{document}

\title{Отчет о выполнении MPI+OpenMP части задания по курсу "Суперкопмьютерное моделирование и технологии"}
\author{Мозговых Василий, 601 группа}
\date{}

\maketitle

\section*{Математическая постановка задачи}

\noindent В треугольнике $D = \{ (x, y)\, |\, y > 0,\, -x+3y-12<0,\, x+3y-12<0\}$, ограниченной кусочно-гладким контуром $\gamma$, рассматривается дифференциальное уравнение Пуассона
\begin{equation}\label{diffur}
  -\Delta u \equiv -\frac{\partial^2 u}{\partial x^2} - \frac{\partial^2 u}{\partial y^2} = 1.
\end{equation}

\noindent Для выделения единственного решения уравнение дополняется граничным условием Дирихле
\begin{equation}\label{granichnoe}
  u(x, y) = 0,\quad (x, y) \in \gamma.
\end{equation}

\noindent Требуется найти функцию $u(x,y)$, удовлетворяющую уравнению (\ref{diffur}) в области $D$ и краевому условию (\ref{granichnoe}) на ее границе.

\section*{Численный метод решения задачи}
Для нахождения приближенного решения задачи (\ref{diffur}), (\ref{granichnoe}) используется метод фиктивных областей.

\subsection*{Метод фиктивных областей}
\noindent Вводится прямоугольник $\Pi = \{ (x, y)\, |\, -3,2 < x < 3,2,\, -0,2 < y < 4.2 \}$ и его граница $\Gamma$. Далее фиксируется малое $\varepsilon > 0$.

\noindent В прямоугольнике $\Pi$ рассматривается следующая задача Дирихле
\begin{equation}\label{diffur-rectangle}
  -\frac{\partial}{\partial x}\bigg{(} k(x,y) \frac{\partial v}{\partial x} \bigg{)} - \frac{\partial}{\partial y} \bigg{(} k(x, y) \frac{\partial v}{\partial y} \bigg{)} = F(x, y),\quad (x,y) \in \Pi \setminus \gamma,
\end{equation}
с краевым условием
$$v(x, y) = 0,\quad (x, y) \in \Gamma,$$
где коэффициент
$$
  k(x,y) = \begin{cases}
    1,\quad (x, y) \in D, \\
    1/\varepsilon,\quad (x, y) \in \Pi \setminus \overline{D},
  \end{cases}
$$
и правая часть уравенения
$$
  F(x,y) = \begin{cases}
    1,\quad (x, y) \in D, \\
    0,\quad (x, y) \in \Pi \setminus \overline{D}.
  \end{cases}
$$

\noindent Требуется найти непрерывную в $\overline{\Pi}$ функцию $v(x,y)$, удовлетворяющую дифференциальному уравнению задачи (\ref{diffur-rectangle}) всюду в $\Pi \setminus \gamma$,
равную нулю на границе $\Gamma$ прямоугольника, и такую, чтобы вектор потока
$$
W(x, y) = -k(x,y)\bigg{(}\frac{\partial v}{\partial x}, \frac{\partial v}{\partial y}\bigg{)}
$$
имел непрерывную нормальную компоненту на общей части границы области $D$ и прямогольника $\Pi$.

\noindent Известно, что функция $v(x,y)$ равномерно приближает решение $u(x,y)$ задачи (\ref{diffur}), (\ref{granichnoe}) в области $D$, а именно
$$
  \max\limits_{(x, y) \in \overline{D}}{|v(x,y) - u(x,y)|} < C \varepsilon,\quad C > 0.
$$

\noindent Это обстоятельство позволяет строить разностные схемы в прямоугольнике $\Pi$ вместо исходной области $D$.

\subsection*{Разностная схема}

\noindent Краевая задача для уравнения (\ref{diffur-rectangle}) решается численно методом конечных разностей. В замыкании прямогольника $\overline{\Pi}$ определяется равномерная сетка
$$
  \overline{\omega}_h = \{ x_i = -3,2 + i h_1,\, i = 0, \dots, M,\quad y_j = -0,2 + j h_2,\, j=0,\dots,N \},
$$
где $h_1=6,4/M$, $h_2=4,4/N$, $h=\max{(h_1, h_2)}$. Через $\omega_h$ обозначим множество узлов сетки, которые не лежат на границе $\Gamma$.

\noindent Рассмотрим линейное пространство $H$ функций, заданных на сетке $\omega_h$.
Обозначим через $w_{ij}$ значение сеточной функции $w \in H$ в узле сетки $(x_i, y_j) \in \omega_h$.
Будем считать, что в пространстве $H$ задано скалярное произведение и $L_2$ норма
$$
(u, v) = \sum\limits_{i=1}^{M-1}\sum\limits_{j=1}^{N-1}{u_{ij} v_{ij} h_1 h_2},\quad \| u \|_{L_2} = \sqrt{(u, u)}.
$$

\noindent В методе конечных разностей дифференциальная задача математической физики заменяется конечно-разностной операторной задачей вида
$$A w = B,$$
где $A: H \rightarrow H$ - оператор, действующий в пространстве сеточных функций, $B \in H$ - известная правая часть.

\noindent Дифференциальное уравнение задачи (\ref{diffur-rectangle}) во всех внутренних точках аппроксимируется разностным уравнением
$$
-\frac{1}{h_1}\bigg{(} a_{i+1 j} \frac{w_{i+1 j} - w_{i j}}{h_1} - a_{i j}\frac{w_{i j} - w_{i-1 j}}{h_1}\bigg{)}
-\frac{1}{h_2}\bigg{(} b_{i j+1} \frac{w_{i j+1} - w_{i j}}{h_2} - b_{i j}\frac{w_{i j} - w_{i j-1}}{h_2}\bigg{)}
= F_{ij},
$$
$$
  i=1,\dots, M-1,\, j=1,\dots,N-1,
$$
в котором коэффициенты
$$
a_{i j} = \frac{1}{h_2}\int\limits_{y_{j-0,5}}^{y+0,5}{k(x_{i-0,5}, \eta) d\eta},\quad b_{i j} = \frac{1}{h_1}\int\limits_{x_{i-0,5}}^{x_{i+0,5}}{k(\xi, y_{j-0,5}) d\xi}
$$
при всех $i=1,\dots,M$, $j=1,\dots,N$. Здесь полуцелые узлы
$$
  x_{i \pm 0,5} = x_i \pm 0,5 h_1,\quad y_{j \pm 0,5} = y_j \pm 0,5 h_2.
$$
Правая часть разностного уравнения
$$
F_{i j} = \frac{1}{h_1 h_2}\iint\limits_{\Pi_{i j}}{F(\xi, \eta) d\xi d\eta},\quad \Pi_{ij} = \{(x,y)\, |\, x_{i-0,5} \le x \le x_{i+0,5},\, y_{j-0,5} \le y \le y_{j+0,5} \}
$$
при всех $i=1,\dots,M-1$, $j=1,\dots,N-1$.

\noindent Краевые условия задачи Дирихле аппроксимируется точно равенством
$$
w_{i j} = w(x_i, y_j) = 0,\quad (x_i, y_j) \in \Gamma.
$$

\noindent Переменные, заданные краевым условием, исключаются из системы уравнений $A w = B$. Все коэффициенты $a_{i j}$, $b_{i j}$ и $F_{i j}$ вычисляются аналитически. В результате нужно решить систему из $(M-1) \times (N-1)$ уравнений с $(M-1) \times (N-1)$ неизвестными.

\subsection*{Метод решения системы линейных алгебраических уравнений}

\noindent Приближенное решение разностной схемы получается итерационным методом сопряженных градиентов. Для ускорения сходимости метода применяется диагональное предобуславливание.

\noindent Пусть оператор $D: H \rightarrow H$ действует на сеточные функции $w \in H$ по правилу
$$
(Dw)_{ij} = [(a_{i+1 j} + a_{i j}) / h_1^2 + (b_{i j+1} + b_{i j}) / h_2^2 ] w_{i j},\quad i=1,\dots,M-1,\, j=1,\dots,N-1.
$$
Этот оператор легко обратим по элементарным формулам, т.к. по сути является поэлементным умножением на положительные числа (в матричном виде это соответствовало бы умножению на диагональную невырожденную матрицу).

\noindent Начальное приближение $w^{(0)}$ к решению разностной схемы выбирается равным нулю во всех точках расчётной сетки.

\noindent Нулевая итерация совершается по формулам скорейшего спуска. Пусть $r^{(0)} = B$ - невязка начального приближения, функция $z^{(0)} = D^{-1} r^{(0)}$. Тогда направление спуска $p^{(1)} = z^{(0)}$, $q^{(1)} = A p^{(1)}$, шаг вдоль направления спуска определяется параметром
$$
  \alpha_{1} = \cfrac{(z^{(0)}, r^{(0)})}{(q^{(1)}, p^{(1)})}.
$$
Следующее приближение $w^{(1)}$ вычисляется согласно равенству
$$
  w^{(1)} = w^{(0)} + \alpha_{1} p^{(1)}.
$$

\noindent Дальнейшие вычисления проводятся по следующим формулам. Пусть выполнено $k$ итераций метода и функции $r^{(k-1)}$, $z^{(k-1)}$, $p^{(k)}$, $q^{(k)}$, $w^{(k)}$, а также коэффициент $\alpha_k$ считаются известными. Тогда невязка последней итерации
$$
  r^{(k)} = r^{(k-1)} - \alpha_k q^{(k)},
$$
сеточная функция $z^{(k)} = D^{-1} r^{(k)}$. Следующее направление спуска
$$
  p^{(k+1)} = z^{(k)} + \beta_{k+1} p^{(k)},
$$
где коэффициент
$$
  \beta_{k+1} = \frac{(z^{(k)}, r^{(k)})}{(z^{(k-1)}, r^{(k-1)})}.
$$
Шаг спуска определяется параметром
$$
\alpha_{k+1} = \frac{(z^{(k)}, r^{(k)})}{(q^{(k+1)}, p^{(k+1)})},
$$
где функция
$$
q^{(k+1)} = A p^{(k+1)}.
$$
Следующее приближение к точному решению $w^{(k+1)}$ вычисляется согласно равенству
$$
  w^{(k+1)} = w^{(k)} + \alpha_{k+1} p^{(k+1)}.
$$

\noindent Метод сопряженных градиентов гарантирует, что при некотором $k$, не превосходящем $(M-1)\times(N-1)$ приближение $w^{(k)}$ станет равным точному решению разностной схемы.
На практике это равенство нарушается из-за ошибок округлений, возникающих в процессе вычислений. Поэтому в качестве условия остановки итерационного процесса помимо максимально числа итераций используется неравенство
$$
\| w^{(k+1)} - w^{(k)} \|_{L_2} < \delta.
$$
Коэффициент $\delta$ выбран равным $3 \cdot 10^{-5}$, исходя из потребности в сходимости метода в отведенное время.

\section*{Работа по созданию MPI программы}

\noindent В MPI программе исходная область разбивается на подобласти (домены), каждая из которых обрабатывается отдельным процессом.
В задании требовалось разработать алгоритм такого разбиения, он реализован следующим образом.
\begin{itemize}
  \item Пусть $n$ - количество процессов, $X$ и $Y$ - размеры глобальной области
  \item Фиксируется $XY_{ratio} = X / Y$
  \item Далее для каждого делителя $n_x$ числа $n$
  \begin{itemize}
    \item $n_y = n \div n_x$
    \item Если $2 n_x < n_y$ или $2 n_y < n_x$, то пропускаем делитель
    \item $x = X \div n_x$, $y = Y \div n_y$ и $xy_{ratio} = x/y$
  \end{itemize}
  \item Среди всех делителей, которые не пропустили, выбираем такой, при котором величина $| xy_{ratio} - XY_{ratio} |$ минимальна
  \item В силу целочисленного деления могли остаться нераспределенные узлы, число которых меньше $n_x \cdot n_y$. Их распределим по "принципу Дирихле" первым процессам по одному.
\end{itemize}

\noindent Указанных в алгоритме условий достаточно, чтобы найденное решение, если оно есть, удовлетворяло требованиям задания. Существование обеспечивается выбором числа процессов.

\noindent После определения оптимального разбиения каждый процесс получает свои локальные параметры сетки - размер сетки и смещение относительно глобальной сетки.
С этими величинами процессы могут независимо инициализировать матрицы коэффициентов.

\noindent Далее код идейно повторяет последовательную версию, за исключением применения пятиточечного шаблона. Здесь необходим обмен граничными значениями с соседними процессами.
Для этого реализована функция \texttt{swap\_values}, которая последовательно осуществляет обмены "снизу вверх", "сверху вниз", "справа налево" и "слева направо".
Здесь посылки реализованы через неблокирующий \texttt{MPI\_Isend}, а приёмы через блокирующий \texttt{MPI\_Recv}, благодаря чему не происходит ситуации взаимной блокировки.
После чтения процессы ожидают конца отправки при помощи \texttt{MPI\_Wait} для синхронизации.

\noindent Также в конце каждого шага требуется вычислять нормы очередного шага. Для этого сначала вычисляется локальная норма, затем при помощи \texttt{MPI\_Allgather} идёт пересылка всем процессам от всех.

\noindent После остановки итераций процессы синхронизируются, идёт пересылка локальных решений процессу с рангом 0. Процесс с рангом 0 записывает локальные решения в глобальную матрицу и выводит результат.

\section*{Работа по созданию MPI+OpenMP программы}

\noindent Основная идея такая же, что и в обычной OpenMP программе - потоки работают независимо до тех пор, пока не возникает обменов информацией.
В данном случае таких обменов всего 3 на каждом шаге:
\begin{enumerate}
  \item Расчёт глобального скалярного произведения через \texttt{MPI\_Allgather}.
  \item Обмен граничными значениями для применения пятиточечного шаблона.
  \item Очередной расчёт скалярного произведения.
\end{enumerate}
Между этими шагами потоки работают с модификатором \texttt{nowait}, который убирает неявный барьер между циклами \texttt{for}.

\noindent Для корректной работы реализации MPI+OpenMP были написаны \texttt{.lsf}-скрипты для привязки потоков OpenMP к ядрам.

\subsection*{Прочее}
\noindent При инициализации коэффициентов $a_{i j}$, $b_{i j}$, $F_{i j}$ соответствующие интегралы вычисляются точно с использованием формулы Ньютона-Лейбница.

\noindent При решении разностной схемы количество итераций ограничено сверху количеством переменных, таким образом, программа гарантированно завершится. Помимо $L_2$-нормы шага метода сопряженных градиентов отслеживаются $L_1$ и $max$ нормы:
$$
  \| u \|_{L_1} = \sum\limits_{i=1}^{M-1}\sum\limits_{j=1}^{N-1}{|u_{ij}| h_1 h_2},\quad \| u \|_{max} = \max\limits_{\substack{i=1,\dots,M-1,\\j=1,\dots,N-1}} {|u_{i j}|}.
$$

\noindent Время работы MPI программы в секундах получается при помощи функции \texttt{MPI\_Wtime}.

\section*{Результаты расчетов}

\noindent В таблице 1 представлены ускорения относительно CPU версии программы для двух расчетных сеток и разного количества процессов MPI и нитей OpenMP.
\begin{table}[h]
  \centering
  \begin{tabular}{|p{2.5cm}|p{3cm}|p{3cm}|p{2cm}|p{2cm}|p{2cm}|}
    \hline
      Количество процессов MPI
        & Количество OpenMP-нитей в процессе
        & Число точек сетки ($M \times N$)
        & Число итераций
        & Время решения в секундах
        & Ускорение \\ \hline
      2 & 1 & 600x400 & 477 & 1.872 & 1.94 \\
      2 & 2 & 600x400 & 477 & 1.373 & 2.65 \\
      2 & 4 & 600x400 & 477 & 0.775 & 4.69 \\
      2 & 8 & 600x400 & 477 & 0.507 & 7.17 \\ \hline
      4 & 1 & 1200x800 & 718 & 5.816 & 4.00 \\
      4 & 2 & 1200x800 & 718 & 4.442 & 5.23 \\
      4 & 4 & 1200x800 & 718 & 2.479 & 9.38 \\
      4 & 8 & 1200x800 & 718 & 1.758 & 13.23 \\ \hline
  \end{tabular}
  \caption{Таблица с результатами расчетов на ПВС IBM Polus (MPI+OpenMP код).}
\end{table}

\noindent Далее изображен график ускорения, построенный по табличным данным.

\begin{figure}[h]
\centering
\includegraphics[width=16cm]{speedup.png}
\end{figure}

\newpage
\noindent Ниже представлен график численного решения задачи на сетке с параметрами $M=1200$, $N=800$.
\begin{figure}[h]
\centering
\includegraphics[width=16cm]{solution_plot.png}
\end{figure}
\end{document}
